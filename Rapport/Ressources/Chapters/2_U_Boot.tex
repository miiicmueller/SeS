% Chapter Template

\chapter{Bootloader U-Boot} % Main chapter title

\label{Chapitre 2} % Change X to a consecutive number; for referencing this chapter elsewhere, use \ref{ChapterX}

\lhead{ \emph{Bootloader U-Boot}} % Change X to a consecutive number; this is for the header on each page - perhaps a shortened title

%----------------------------------------------------------------------------------------
%	SECTION 1
%----------------------------------------------------------------------------------------
\section{MMC partitionning}

\subsection{Automated MMC partitionning}

Le tout premier laboratoire était destiné à créer un petit script qui doit permettre d'automatiser la procédure d'installation de la MMC. Voilà directement notre proposition de script :

\begin{lstlisting}[frame=single,style=Console]  % Start your code-block

#!/bin/bash

diskId=$1
specCmd=$2

echo  "--------------- ODROID XU-3 MMC partition utility --------------"
echo  "| 	Written by : Michael Mueller                           |"
echo  "| 	Modified by : Cyrille Savy                             |"
echo  "----------------------------------------------------------------"


# Verification que l'utilisateur est bien root

uid=$(whoami)

if [ "$diskId" == "" ]
then
	echo "./flash_disk.sh <dev_name> [erase] !"
	exit -1
else 	echo  "Writing on device : $diskId" 
fi


if [ "$uid" == "root" ]
then
	if [ "$specCmd" == "erase_all" ]
	then
		echo "Erasing all..."
		dd if=/dev/zero of=$diskId bs=512 seek=1 count=2097152
		sync
	else echo "Writing partition only..."
	fi

	echo "Creating msdos MBR..."
	# First sector: msdos
	parted $diskId mklabel msdos

	echo "Creating bootfs..."
	# create bootfs 64MB
	parted $diskId mkpart primary ext4 131072s 262143s

	echo "Creating rootfs..."
	# create rootfs 256MB
	parted $diskId mkpart primary ext4 262144s 786431s

	echo "Creating usrfs..."
	# create usrfs  256MB
	parted $diskId mkpart primary ext4 786432s 1310719s  
	
	echo "Formatting bootfs..."
	# format with label
 	mkfs.ext4 $diskId"1" -L bootfs

	echo "Formatting rootfs..."
	# format with label
 	mkfs.ext4 $diskId"2" -L rootfs

	echo "Formatting usrfs..."
	# format with label
 	mkfs.ext4 $diskId"3" -L usrfs
	sync

	echo "Copying third-party firmware..."
	#copy firmware & bl1.bin, bl2.bin,tzsw.bin
	dd if=~/workspace/xu3/buildroot/output/images/xu3-bl1.bin of=$diskId bs=512 seek=1  
	dd if=~/workspace/xu3/buildroot/output/images/xu3-bl2.bin of=$diskId bs=512 seek=31 
	dd if=~/workspace/xu3/buildroot/output/images/xu3-tzsw.bin of=$diskId bs=512 seek=719 
	sync

	echo "Copy U-BOOT..."
	#copy u-boot
	dd if=~/workspace/xu3/buildroot/output/images/u-boot.bin of=$diskId bs=512 seek=63 
	sync

	echo "Copy kernel and dtb..."
	mkdir /mnt/bootfs
	mount -t ext4 $diskId"1" /mnt/bootfs
	#copy kernel & flattened device tree
	cp ~/workspace/xu3/buildroot/output/images/uImage /mnt/bootfs/
	cp ~/workspace/xu3/buildroot/output/images/exynos5422-odroidxu3.dtb /mnt/bootfs/ 
	sync
	#unmount filesystem
	umount /mnt/bootfs

	echo "Copy rootfs..."
 	#copy rootfs
	dd if=~/workspace/xu3/buildroot/output/images/rootfs.ext4 of=$diskId"2" bs=512 
	#dd if=~/workspace/xu3/buildroot/output/images/rootfs.ext4 of=$diskId bs=512 seek=262144 
	sync

	echo ""
	echo "DONE WITH SUCCESS! "
	echo ""
	exit 0

else echo "root permission required !";exit -1
fi

\end{lstlisting}

Voici quelques explication pour les grandes lignes de ce scripts : dans premier temps on vérifie que l'on a passé un "device" en paramètre (par exemple "/dev/sdb"). Ensuite vérifie que l'utilisateur possède les droits suffisant ("root"). 
%TODO : Ajouter la suite des explications

\pagebreak
\subsection{MMC File positionning}

Dans cette partie du laboratoire nous avons du essayer de déplacer les divers fichiers sur la carte SD. Nous avons commencé par déplacer "bl1.bin" qui est un premier bootloader. Voilà les commandes utilisées (effacement de la mémoire et réécriture dans une autre zone) :
\begin{lstlisting}[frame=single,style=Console]  % Start your code-block

# BL1
dd if=/dev/zero of=$diskId bs=512 seek=1 count=31
dd if=~/workspace/xu3/buildroot/output/images/xu3-bl1.bin of=$diskId bs=512 seek=17849  # déplacé
\end{lstlisting}

La carte ne fait rien! Évidemment, si le processeur ne trouve pas le "bootloader", il ne peut charger aucun logiciel. 
Nous avons ensuite essayer de déplacer "bl2.bin". Voilà les commandes utilisées (effacement de la mémoire et réécriture dans une autre zone) : 
\begin{lstlisting}[frame=single,style=Console]  % Start your code-block

# BL2
dd if=/dev/zero of=$diskId bs=512 seek=31 count=29
dd if=~/workspace/xu3/buildroot/output/images/xu3-bl2.bin of=$diskId bs=512 seek=17849 # déplacé
\end{lstlisting}

La carte ne fait rien non plus... On peut partir du principe que le premier bootloader à été chargé en mémoire et exécuté et que le deuxième n'étant pas présent n'as pas démarré. Donc le système ne fonctionne pas. 
Nous avons essayé de déplacer la "trustzone". Voilà les commandes utilisées (effacement de la mémoire et réécriture dans une autre zone) : 
\begin{lstlisting}[frame=single,style=Console]  % Start your code-block

#TZ
dd if=/dev/zero of=$diskId bs=512 seek=719 count=512
dd if=~/workspace/xu3/buildroot/output/images/xu3-tzsw.bin of=$diskId bs=512 seek=17849 # déplacé
\end{lstlisting}

Cette fois u-boot démarre. Est-ce normal ou est-ce que la "trust-zone" n'est pas utilisée, nous ne savons pas... Peut-être que nous avons fait des erreurs de manipulations. Essayons encore de déplacer u-boot. 
Voilà les commandes utilisées (effacement de la mémoire et réécriture dans une autre zone) : 

\begin{lstlisting}[frame=single,style=Console]  % Start your code-block

#Uboot
dd if=/dev/zero of=$diskId bs=512 seek=63 count=552
dd if=~/workspace/xu3/buildroot/output/images/u-boot.bin of=$diskId bs=512 seek=17849 # déplacé
\end{lstlisting}

U-boot ne démmare pas (on peut en déduire que c'est logique...)! Par contre le ventilateur c'est mis à tourner, ce qui porte à penser que "bl2.bin" à bien démarré, mais n'as pas trouvé "u-boot".

%TODO Cyrille : à ajouter tes parties


\section{U-Boot hardening}

\subsection{GCC compilation command line}

\subsubsection{Debug symbol stripping}

\subsubsection{Using canaries}
